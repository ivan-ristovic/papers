\documentclass{beamer}
\usepackage[utf8]{inputenc} % make weird characters work
\usepackage[english,serbian]{babel}
%\usetheme{Pittsburgh}
%\usecolortheme{beetle}
\setbeamertemplate{navigation symbols}{}

\usepackage{listings}
\definecolor{codegreen}{rgb}{0,0.6,0}
\definecolor{codegray}{rgb}{0.5,0.5,0.5}
\definecolor{codeblue}{rgb}{0.0,0,0.82}
\lstdefinestyle{mystyle}{
    numbers=left,
    numberstyle=\scriptsize,
    numbersep=8pt,
    commentstyle=\color{codegray},
    keywordstyle=\color{codegreen},
    numberstyle=\tiny\color{codeblue},
    stringstyle=\color{codegreen},
    basicstyle=\ttfamily\footnotesize,
    breakatwhitespace=false,
    breaklines=true,
    captionpos=b,
    keepspaces=true,
    showspaces=false,
    showstringspaces=false,
    showtabs=false,
    tabsize=4,
    xleftmargin=3em,
    framexleftmargin=1.5em
}
\lstset{style=mystyle}

\usepackage[font=scriptsize,labelfont=bf]{caption}

\title{Neograničena provera modela softvera kroz inkrementalno SAT rešavanje}
\author{Ivan Ristovi\'c}
\date{}


\begin{document}
\begin{frame}
    \titlepage
\end{frame}

\begin{frame}{Uvod}
    \begin{itemize}
        \item Ograničena provera modela softvera
        \item Inspiracija za neograničenu proveru modela softvera
        \item Nov način kodiranja stanja programa omogućava uklanjanje granice koja postoji kod ograničene provere modela softvera
        \item Kodiranje stanja programa u \emph{DimSpec} formulu korišćenjem alata \emph{LLUMC}
        \item Korišćenje inkrementalnih SAT rešavača za nalaženje modela
    \end{itemize}
\end{frame}

\begin{frame}{Opis rada neograničene provere modela}
    \begin{itemize}
        \item Sekvencijalni C kod
        \item Prevođenje u \emph{LLVM} međureprezentacija C koda
        \item Kodiranje stanja programa u SMT formule
        \item Stanje greške predstavlja ciljno stanje
        \item Prevođenje SMT formule u \emph{DimSpec} formulu (SAT)
        \item Korišćenje inkrementalnih SAT rešavača za nalaženje modela ili algoritme za nalaženje invarijanti (\emph{IC3})
        \item Model se koristi kao dokaz neispravnosti programa
    \end{itemize}
\end{frame}

\begin{frame}{LLVM reprezentacija}
    \begin{itemize}
        \item \emph{LLVM} je open-source compiler framework projekat.
        \item Za neograničenu proveru modela se koristi \emph{LLVM} međureprezentacija C koda koja se naziva \emph{LLVM modul}.
        \item \emph{LLVM} modul se može grupisati u nekoliko jedinica:
            \begin{itemize}
                \item instrukcija
                \item osnovni blok
                \item funkcija
            \end{itemize}
    \end{itemize}
\end{frame}

\begin{frame}{Prostor stanja programa}
    \begin{itemize}
        \item Stanja programa se posmatraju kao stanja svakog osnovnog bloka zasebno. Slično je i za tranzicije između stanja.
        \item Jedno stanje se sastoji od promenljivih.
        \item Svaka promenljiva se kodira kao bit-vektor dužine \emph{n}. Svi bitovi zasebno se posmatraju kao promenljive $x_{1},  \dots , x_{n}$.
        \item Uvodimo dva specijalna stanja:
            \begin{itemize}
                \item \emph{ok} stanje - iz ovog stanja ne mogu nastati greške
                \item \emph{error} stanje
            \end{itemize}
    \end{itemize}
\end{frame}

\begin{frame}{Prostor stanja programa}
    \begin{itemize}
        \item Ako je $|B|$ broj osnovnih blokova u programu, onda je  $\lceil\log_{2}{|B| + 2}\rceil$ bitova potrebno za enkodiranje jednog osnovnog bloka.
        \item Promenljivu koja kodira trenutni blok označavamo sa \emph{curr}.
        \item S obzirom da vrednosti promenljivih mogu da zavise od vrednosti u prethodnom bloku, uvodimo promenljivu \emph{prev} koja kodira prethodni blok.
    \end{itemize}
\end{frame}

\begin{frame}{DimSpec formula}
    \begin{itemize}
        \item Svako stanje $t_{0}, \dots , t_{k}$ sadrži vrednosti promenljivih u tekućem osnovnom bloku.
        \item \emph{DimSpec} formula $\mathcal{F}$ se sastoji od četiri SMT formule: $\mathcal{I}$, $\mathcal{U}$, $\mathcal{G}$ i $\mathcal{T}$.
        \item $\mathcal{I}$ predstavlja skup inicijalnih klauza, tj. onih klauza koje stanje $t_{0}$ zadovoljava.
        \item $\mathcal{G}$ predstavlja skup završnih klauza, tj. onih koje stanje $t_{k}$ zadovoljava.
        \item $\mathcal{U}$ predstavlja skup klauza koje su zadovoljene stanjima $t_{i}$.
        \item $\mathcal{T}$ predstavlja skup klauza koje su zadovoljene parovima stanja $t_{i}, t_{i+1}$.
    \end{itemize}
\end{frame}

\begin{frame}{DimSpec formula}
    \begin{itemize}
        \item Testiranje da li je završno stanje dostižno iz početnog u \emph{k} koraka, je ekvivalentno ispitivanju zadovoljivosti formule $F_{k}$:
        \begin{displaymath}
            F_{k} = \mathcal{I}(0) \wedge \Bigg( \bigwedge_{i=0}^{k-1}{(\mathcal{U}(i) \wedge \mathcal{T}(i, i+1))} \Bigg) \wedge \mathcal{U}(k) \wedge \mathcal{G}(k)
        \end{displaymath}
        \item Jedan od načina da se nađe najmanji broj koraka za koje se završno stanje dostiže iz polaznog je da se rešavaju formule $F_{1}, F_{2}, \dots$ sve dok se ne nađe zadovoljiva formula.
        \item Efikasan način da se ovo implementira je koriščenjem \emph{Incremental SAT-Solving-a}.
    \end{itemize}
\end{frame}

\begin{frame}{Prevođenje SMT u SAT}
    \begin{itemize}
        \item Enkodiranje opisano iznad nam kao rezultat daje četiri SMT formule
        \item \emph{bit-blasting}
        \item \emph{LLUMC} automatski vrši ovu transformaciju
    \end{itemize}
\end{frame}

\begin{frame}{Inkrementalno SAT rešavanje}
    \begin{itemize}
        \item Inkrementalno dodavanje klauza tokom rešavanja problema zadovoljivosti
        \item Funkcije \emph{add(C)} i \emph{solve(A)}, gde je \emph{C} klauza a \emph{A} skup literala koji se nazivaju pretpostavke (engl. \emph{assumptions})
        \item Klauze se dodaju koriščenjem funkcije \emph{add} i njihova konjunkcija se rešava pod predpostavkom da su vrednosti svih literala iz \emph{A} tačne, koristeći \emph{solve(A)}
        \item Proces inkrementalnog SAT rešavanja za \emph{DimSpec} formule:
            $$\begin{array}{rcl}
            step(0) & : & \texttt{add}(\mathcal{I}(0) \wedge (a_{0} \vee \mathcal{G}(0)) \wedge \mathcal{U}(0)) \\
                &   & \texttt{solve}(assumptions = \neg{a_{0}}) \\
            step(k) & : & \texttt{add}(\mathcal{T}(k - 1, k) \wedge (a_{k} \vee \mathcal{G}(k)) \wedge \mathcal{U}(k)) \\
                &   & \texttt{solve}(assumptions = \neg{a_{k}})
            \end{array}$$
    \end{itemize}
\end{frame}

\begin{frame}{Ograničenja Unbounded Software Model Checking-a}
    \begin{itemize}
        \item Šta je uopste greška u softveru?
        \item Nemoguće je pokriti sve moguće greške.
        \item Stoga se u nastavku ograničavamo na \emph{prekoračenja}.
        \item Koristićemo funkcije \texttt{assume} i \texttt{assert}.
        \item Program se ponaša po specifikaciji ukoliko svi pozivi funkcije \texttt{assert} vrate \texttt{true} pod uslovom da su svi uslovi iz poziva \texttt{assume} zadovoljeni.
    \end{itemize}
    \newtheorem{definicija}{Definicija (Greška programa za alat LLUMC)}
    \begin{definicija}
        Neka je p program. Greška u programu p postoji ukoliko su svi pozivi funkcije \textbf{assume} pre poziva funkcije \textbf{assert} ili mogućeg prekoračenja vratili \textbf{true} i važi jedno od navedenog:
        \begin{enumerate}
            \item Funckija \textbf{assert} je vratila \textbf{false}.
            \item Desilo se prekoračenje prilikom izvršavanja aritmetičke operacije.
        \end{enumerate}
    \end{definicija}
\end{frame}

\begin{frame}[fragile]{Primer}
    \begin{figure}[!h]
        \centering
\begin{lstlisting}[language=C]
int main(void)
{
    unsigned int x = 4294967295-101;
    while (x >= 10) {
        x += 2;
    }
    __VERIFIER_assert(x % 2);
}
\end{lstlisting}
    \end{figure}
\end{frame}

\begin{frame}[fragile]{Primer - LLVM međureprezentacija}
    \begin{figure}[!h]
        \centering
\begin{lstlisting}[basicstyle=\tiny]
define i32 @main() #0 {
%1 = alloca i32, align 4
%x = alloca i32, align 4
store i32 0, i32* %1
store i32 4294967194, i32* %x, align 4
br label %2

; <label>:2                            ; preds = %5, %0
%3 = load i32, i32* %x, align 4
%4 = icmp uge i32 %3, 10
br i1 %4, label %5, label %8
12
; <label>:5                            ; preds = %2
%6 = load i32, i32* %x, align 4
%7 = add i32 %6, 2
store i32 %7, i32* %x, align 4
br label %2
18
; <label>:8                            ; preds = %2
%9 = load i32, i32* %x, align 4
%10 = urem i32 %9, 2
call void @__VERIFIER_assert(i32 %10)
%11 = load i32, i32* %1
t i32 %11
}
\end{lstlisting}
\end{figure}
\end{frame}

\begin{frame}[fragile]{Primer - Optimizovana LLVM međureprezentacija}
    \begin{figure}[!h]
        \centering
\begin{lstlisting}[basicstyle=\tiny]
define i32 @main() #0
entry:
%tmp = icmp uge i32 4294967194, 10
br i1 %tmp, label %bb1, label %return

bb1:                           ; preds = %entry, %bb1
%x.0 = phi i32 [ %tmp2, %bb1 ], [ 4294967194, %entry ]
%tmp2 = add i32 %x.0, 2
%tmp3 = icmp uge i32 %tmp2, 10
br i1 %tmp3, label %bb1, label %return

return:                        ; preds = %bb1, %entry
%x.1 = phi i32 [ 4294967194, %entry ], [ %tmp2, %bb1 ]
%tmp4 = urem i32 %x.1, 2
%tmp.i = icmp ne i32 %tmp4, 0
br i1 %tmp.i, label %__VERIFIER_assert.exit, label %bb1.i

bb1.i:                         ; preds = %return
call void bitcast (void (...)* @__VERIFIER_error to void ()*)() #3
unreachable

__VERIFIER_assert.exit:        ; preds = %return
ret i32 0
\end{lstlisting}
    \end{figure}
\end{frame}

\begin{frame}{Primer - Kodiranje stanja}
    $$entry \rightarrow 1$$
    $$bb1.lr.ph \rightarrow 2$$
    $$bb1 \rightarrow 3$$
    $$bb.return\_crit\_edge \rightarrow 4$$
    $$return \rightarrow 5$$
    $$bb.1 \rightarrow 6$$
    $$\_\_VERIFIER\_assert\_exist \rightarrow 7$$
    $$ok \rightarrow 8$$
    $$error \rightarrow 9$$
\end{frame}

\begin{frame}{Primer - Kodiranje skupova $\mathcal{I}, \mathcal{G}, \mathcal{U}$}
$$\mathcal{I} = \{ curr = 1 \wedge prev = 1 \}$$
$$\mathcal{G} = \{ curr = 9 \}$$
$$\mathcal{U} = \{ curr <= 9 \wedge prev <= 9 \}$$
\end{frame}

\begin{frame}{Primer - Kodiranje skupa $\mathcal{T}$}
    $
    \displaystyle
    \small
    \begin{array}{cl}
    \mathcal{T} = \{ & \\
    & (curr = 1 \wedge prev = 1 \Rightarrow curr' = 9 \wedge prev' = 9 \wedge tmp2' = tmp2) \\
    \wedge & (curr = 2 \Rightarrow curr = 3 \wedge prev' = 2 \wedge tmp2' = tmp2) \\
    \wedge & (curr = 3 \wedge 10 \leq (2 + ite(prev = 2, 10, tmp2)) \\
    & \Rightarrow curr' = 3 \wedge prev' = 3 \wedge tmp2' = (2 + ite(prev = 2, 10, tmp2)))\\
    \wedge & (curr = 3 \wedge 10 > (2 + ite(prev = 2, 10, tmp2)) \\
    & \Rightarrow curr' = 4 \wedge prev' = 3 \wedge tmp2' = (2 + ite(prev = 2, 10, tmp2)))\\
    \wedge & (curr = 4 \Rightarrow curr' = 5 \wedge prev' = 4 \wedge tmp2' = tmp2) \\
    \wedge & (curr = 5 \wedge 0 \neq bvmod(ite(prev = 4, tmp2, 10), 2) \\
    & \Rightarrow curr' = 7 \wedge prev' = 5 \wedge tmp2' = tmp2)\\
    \wedge & (curr = 5 \wedge 0 = bvmod(ite(prev = 4, tmp2, 10), 2) \\
    & \Rightarrow curr' = 6 \wedge prev' = 5 \wedge tmp2' = tmp2)\\
    \wedge & (curr = 6 \Rightarrow curr' = 9 \wedge prev' = 6 \wedge tmp2' = tmp2) \\
    \wedge & (curr = 7 \Rightarrow curr' = 8 \wedge prev' = 7 \wedge tmp2' = tmp2) \\
    \wedge & (curr = 8 \Rightarrow curr' = 8 \wedge prev' = 8 \wedge tmp2' = tmp2) \\
    \wedge & (curr = 9 \Rightarrow curr' = 9 \wedge prev' = 7 \wedge tmp2' = tmp2) \\
    \} &
    \end{array}
    $
\end{frame}

\begin{frame}{Pitanja}
    \centering
    ???
\end{frame}


\end{document}
